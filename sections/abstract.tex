India faces a severe shortage of healthcare specialists, especially in the departments of ENT and dermatology. More than 23,000 medical posts sanctioned, as of March 2023, are vacant across India. More than 9,000 vacancies are in primary health centers (PHCs) alone. According to the National Health Profile 2019, India has only about 1 doctor per 1,457 people, whereas the WHO recommends 1:1,000. The shortage is even more acute in rural areas, where around 70\% of India's population resides. Among the more serious consequences of access barriers are delays in diagnosis, complication severity, and increased treatment cost. We propose a smart telemedicine system, which can perform dermatoscopic and pharyngoscopic imaging with a portable device consisting of a camera and microcontroller. The images captured are uploaded through a mobile app and sent to a backend server where the model performs preprocessing and AI-based diagnostic inference. The images captured are uploaded through a mobile app and sent to a backend server where the model performs preprocessing and AI-based diagnostic inference. The results are then transmitted back to the app, allowing healthcare workers and patients in underserved areas to access specialist-level diagnostic support remotely. Being designed for use in remote regions, the system equips primary care providers with sophisticated diagnostic tools that improve healthcare delivery without major infrastructure. 
